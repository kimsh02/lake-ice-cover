\documentclass[12pt]{report}

\usepackage{fullpage}
\usepackage{amsmath,amssymb,bm,upgreek,mathrsfs}
\usepackage{algorithmic,algorithm}
\usepackage{graphicx,subcaption}
\usepackage{setspace}
\usepackage{color}
\usepackage{multirow}
\usepackage{alltt}
\usepackage{cancel}
\usepackage{listings}

\doublespacing

\DeclareMathOperator*{\argmax}{arg\,max}
\DeclareMathOperator*{\argmin}{arg\,min}

\newcommand{\N}{\mathcal{N}} \newcommand{\U}{\mathcal{U}}
\newcommand{\Poi}{{\text Poisson}} \newcommand{\Exp}{{\text Exp}}
\newcommand{\G}{\mathcal{G}} \newcommand{\Ber}{{\text Bern}}
\newcommand{\Lap}{{\text Laplace}} \newcommand{\btheta}{\boldsymbol{\theta}}
\newcommand{\bSigma}{\boldsymbol{\Sigma}}

\newcommand{\E}[1]{\mathbb{E}[#1]}
\newcommand{\Cov}[2]{\mathbb{C}\mathrm{ov}(#1,#2)}

\def\*#1{\mathbf{#1}} \newcommand*{\V}[1]{\mathbf{#1}}

%%%%%%%%%%%%%%%%%%%%%%%%%%%%%%%%%%%%%%%%%%%%%%%%%%%%%%%%%%%%%%%%%%%%%%

\begin{document}

\centerline{\it CS 480 HW \#2}

\begin{enumerate}

\item[1.] Effort level.

\item[a.] The homework took me 12 hours.
\item[b.] N\textbackslash A
\item[c.] N\textbackslash A

\item[2.] Paired t-test.

\item[a.] 95\% confidence interval.
\item[b.] Two-sided paired t-test for null-hypothesis.
\item[c.] Report p-value for testing done in b.
\item[d.] Describe pseudocode for bootstrapping method for this problem.
\item[e.] Implement and perform bootstrapping paired test with computed p-value.

\item[3.] Linear regression.

\item[a.] Plot year versus ice days.
\item[b.] Split the datasets into training and testing. Compute std and mean for
  the two lakes respetively.
\item[c.] Using training sets, train a linear regression model.
\item[d.] Mean squared error on the test set.
\item[e.] Train a linear regression model using Monona.
  \begin{itemize}
  \item[i.] Interpret the sign of $\gamma_1$.
  \item[ii.] Assessing a viewpoint formed from the model.
  \end{itemize}

\item[4.] Maximum likelihood estimation for linear regression.

\item[a.] Likelihood of the ith datapoint.

  $Laplace(\mu ,b) \rightarrow f(z|\mu , b) = \frac{1}{2b}e^{-\frac{|z-u|}{b}}$

  $Laplace(w^Tx,1) \rightarrow f(z|w^Tx, 1) = \frac{1}{2}e^{-|z-u|}$

  $P(y_i|x_iw)=f(y_i|w^Tx_i,1)$

  $= \frac{1}{2}e^{-|y_i-w^Tx_i|}$
\item[b.] Log-likelihood of the dataset.
\item[c.] Maximum likelihood estimator $w$ under the above model.

% \item[B.] Making a world coordinate system.
%   Below in Figure 4, I provide the world coordinate system for the image provided
%   for hw03.

%   \begin{figure}[H]
%     \centering \fbox{
%       \begin{minipage}{.98\linewidth}
%         \includegraphics[width=\linewidth]{IMG_0862.jpeg}
%         \caption{World coordinate system for provided image for hw03. The X, Y,
%           and Z axes are clearly labeled, and the direction of the axes' arrows
%           indicate in which direction they grow. 15 blue points have been
%           selected for this experiement as shown, and their (X,Y,Z) coordinates
%           are labeled in yellow.}
%         \label{fig:b1}
%       \end{minipage}}
%   \end{figure}

%   These 15 blue world coordinate points in Figure 4 will be used for Part C to
%   compare how two cameras map these 3D world coordinates to 2D. The files
%   world\_coords.txt and image\_coords.txt have also been provided in this
%   submission for Part C. The world coordinates of the points are as they are in
%   Figure 4, and the image coordinates (X, Y) were obtained using the
%   `datacursormode on' feature in MATLAB and manually clicking on the labeled
%   blue points.

% \item[C.] Analyzing how a camera matrix maps points from 3D to 2D.

%   Below in Figure 5, I provide the visual representation of our clicked world
%   coordinates along with the two camera estimates of our world coordinates.

%   \begin{figure}[H]
%     \centering \fbox{
%       \begin{minipage}{.98\linewidth}
%         \includegraphics[width=\linewidth]{Screenshot 2024-09-29 at 11.19.16 PM.png}
%         \caption{The red clicked points are our world coordinates from Figure 4
%           moved onto this black canvas to compare to our first (green) and
%           second (blue) camera estimates. Produced in MATLAB.}
%         \label{fig:c1}
%       \end{minipage}}
%   \end{figure}

%   In Figure 5 above, we can see that both the first and camera estimates are
%   visually close to our red clicked points. One thing to note when plotting the
%   estimates was to flip the X and Y coordinates generated from the given camera
%   matrices which we explored the reason for in Part A.

%   The RMSE of the distance between the camera estaimte point and the actual
%   clicked point was calculated for both cameras in MATLAB. Again, when
%   calculating the distsances between points, we have to flip the coordinates
%   generated from the camera matrices, or flip the image coordinates of the world
%   coordinates. The key thing is to be consistent with our coordinate system
%   (either work in X vs Y values or work with indices of a matrix). The RMSEs are
%   as follows:

%   $RMSE_{camera1}=23.9$

%   $RMSE_{camera2}=40.7$

%   The RMSE of the camera estimated points and actual world coordinate points is
%   also known as the re-projection error as stated in the homework. From the
%   reported RMSEs of the cameras, the first camera is better quantitatively
%   speaking, and from scrutinizing Figure 5, one can see that the first camera
%   (green) points are closer to the clicked (red) points than the second (blue)
%   camera points on average.

%   Thus, the first camera's matrix is more accurate.

\end{enumerate}

\end{document}
